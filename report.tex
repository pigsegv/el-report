\documentclass{ieeeaccess}
\usepackage{cite}
\usepackage{amsmath,amssymb,amsfonts}
\usepackage{algorithmic}
\usepackage{graphicx}
\usepackage{textcomp}
\usepackage{threeparttable}

\usepackage{bm}
\makeatletter
\AtBeginDocument{\DeclareMathVersion{bold}
\SetSymbolFont{operators}{bold}{T1}{times}{b}{n}
\SetSymbolFont{NewLetters}{bold}{T1}{times}{b}{it}
\SetMathAlphabet{\mathrm}{bold}{T1}{times}{b}{n}
\SetMathAlphabet{\mathit}{bold}{T1}{times}{b}{it}
\SetMathAlphabet{\mathbf}{bold}{T1}{times}{b}{n}
\SetMathAlphabet{\mathtt}{bold}{OT1}{pcr}{b}{n}
\SetSymbolFont{symbols}{bold}{OMS}{cmsy}{b}{n}
\renewcommand\boldmath{\@nomath\boldmath\mathversion{bold}}}
\makeatother

\def\BibTeX{{\rm B\kern-.05em{\sc i\kern-.025em b}\kern-.08em
    T\kern-.1667em\lower.7ex\hbox{E}\kern-.125emX}}

%Your document starts from here ___________________________________________________
\begin{document}
\history{Date of publication xxxx 00, 0000, date of current version xxxx 00, 0000.}
\doi{10.1109/ACCESS.2024.0429000}

\title{Increasing the Strength of Pervious Concrete While Maintaining 
       Permability}
\author{
    \uppercase{Pranav V}\authorrefmark{1}, \uppercase{Abhay V}\authorrefmark{1}, 
    \uppercase{Suyash J}\authorrefmark{1}, \uppercase{Soham S}\authorrefmark{2},
    \uppercase{Rakshith P}\authorrefmark{3}
}

\address[1]{Department of Electronics and Communication Engineering,
            RV College of Engineering, Bangalore, India}
\address[1]{Department of Mechanical Engineering,
            RV College of Engineering, Bangalore, India}
\address[1]{Department of Information Science and Engineering,
            RV College of Engineering, Bangalore, India}

\markboth
{Author \headeretal: Preparation of Papers for IEEE TRANSACTIONS and JOURNALS}
{Author \headeretal: Preparation of Papers for IEEE TRANSACTIONS and JOURNALS}

\corresp{Corresponding author: Pranav V (e-mail: pranavvv.ec24@rvce.edu.in).}


\begin{abstract}
Pervious concrete is a sustainable construction material developed to mitigate
urban issues such as stormwater runoff, reduced groundwater recharge, and
surface flooding. This study focuses on the design and experimental
evaluation of a pervious concrete mix aimed at achieving adequate
mechanical strength while ensuring desirable permeability. The
investigation involved selecting suitable aggregates, determining an
appropriate water–cement ratio, and incorporating admixtures to enhance
overall performance. Compressive strength, tensile strength, and
permeability tests were conducted to assess the mix’s applicability in
pavements and low-traffic areas. The results demonstrated that the proposed
mix design offers a practical balance between structural integrity and
permeability, highlighting its potential as an environmentally friendly
alternative to conventional paving materials. Additionally, the paper
discusses challenges encountered during mix design and implementation and
suggests directions for future research to enable its effective use in
large-scale applications.
\end{abstract}

\begin{keywords}
    Pervious concrete, Sustainable Engineering, 
\end{keywords}

\titlepgskip=-21pt

\maketitle

\section{Introduction}
\label{sec:introduction}
\PARstart{R}{apid} urbanization has led to extensive construction of impervious 
surfaces such as asphalt and conventional concrete pavements, which disrupt the 
natural hydrological cycle. These surfaces prevent water infiltration, resulting 
in increased surface runoff, urban flooding, and reduced groundwater recharge. 
In response to these environmental concerns, there has been a growing interest 
in sustainable construction materials that support stormwater management. One 
such material is pervious concrete, a special type of concrete with a high void 
content that allows water to pass through its structure. 

Pervious concrete is composed of coarse aggregates, cement, water, and little to 
no fine aggregates. Its interconnected pore network enables infiltration of 
rainwater, making it suitable for sidewalks, parking lots, driveways, and 
low-traffic roads. In addition to hydrological benefits, pervious concrete can 
reduce the urban heat island effect, improve skid resistance, and contribute 
toward LEED (Leadership in Energy and Environmental Design) credits in green 
building certification systems. 

Despite its advantages, the widespread use of pervious concrete has been limited 
due to challenges in achieving an optimal balance between permeability and 
mechanical strength. In this study, we focus on the effects of the  water-cement 
ratio on the physical properties of pervious concrete. To this end, two batches 
of 6 pervious concrete cylinders, with the second batch having a lower 
water-cement ratio, were made, and tested for compresssive strength, split 
tensile strength, and permeability. Superplasticizer (SP) was used to increase 
the workability of the mixes made with the second recipe. 



\section{Experimental Setup}
Two batches of 6 cylinders each were cast. Common to both batches were the
cementitious material which was a mixture of cement and fly-ash in the ratio of 
4:1 by mass, coarse aggregates whose sizes ranged from 4.75-9.5 mm, 
and polypropylene fibres (PPF). 
\
\begin{table}
    \begin{threeparttable}
        \caption{\textbf{Mix Design}}
        \label{table:mix-design}
        \setlength{\tabcolsep}{20pt}
        \def\arraystretch{1.5}%
        \begin{tabular}{ l r  r }
            \hline
            & \multicolumn{1}{c}{Batch 1} & \multicolumn{1}{c}{Batch 2} \\
            \hline
            Cement, $kg/m\textsuperscript{3}$  & 280  & 280  \\
            Coarse Aggregate, $kg/m\textsuperscript{3}$ & 1420 & 1420 \\
            Fly-ash, $kg/m\textsuperscript{3}$ & 70   & 70   \\
            Water, $kg/m\textsuperscript{3}$   & 119  & 95.2 \\
            SP, $\%$\textsuperscript{1}        & ---  & 0.5  \\
            PPF, $\%$\textsuperscript{2}       & 0.2  & 0.2  \\
            \hline
        \end{tabular} 
        \begin{tablenotes}
            \item[1] \footnotesize of cementitious material
            \item[2] \footnotesize of aggregates
        \end{tablenotes}
    \end{threeparttable}
\end{table}
\
The proportions for the rest of the materials used are given in table 
\ref{table:mix-design}.

\section{Testing Method}
The samples were tested after 14 and 28 days for tensile and compressive 
strenghts, and after 28 days for permeability.

\subsection{Compressive Strength}
The sample was placed in the universal testing machine (UTM) on one of its 
circular faces, and compressed. The pace rate was set to 1.8 KN/s. 
The compressive strength $f_c$ is calculated with the formula 
\[f_c = \frac{P}{A}\] where $P$ is the maximum force on the sample, and $A$ is
the area over which the force is applied.

\subsection{Tensile Strength}
The tensile strength was found using the Brazilian test, in which, the sample
was placed horizontally, in between two metal bars oriented parallel to the axis
of the sample, inside the UTM. The tensile strength $f_t$ was calculated with
the formula \[f_t = \frac{2P}{\pi LD}\] where $P$ is the force at the point of
failure, and $L$ and $D$ are respectively the length and the diameter of the
sample.

\subsection{Permeability}
The tensile strength was found using a makeshift falling head permeameter.
By measuring the time taken for the head to move from a height $h_1$ down to
$h_2$, the permeability $k$ can be calculated using the formula 
\[k = \frac{aL}{At} \ln{\frac{h_1}{h_2}}\] where $a$ is the cross-sectional area
of the standpipe, $t$ is the time taken for the head to fall from $h_1$ to 
$h-2$, and $L$ and $A$ are the dimensions of the sample.


\section{Results}

\begin{table}[htb]
    \begin{threeparttable}
        \caption{\textbf{Compressive and Tensile Strength}}
        \label{table:str-test}
        \setlength{\tabcolsep}{16.5pt}
        \def\arraystretch{1.5}%
        \begin{tabular}{ l r r r r }
            \hline
            & \multicolumn{2}{c}{Compressive, $Mpa$} & 
                \multicolumn{2}{c}{Tensile, $Mpa$} \\

            \cline{2-5}
            & \multicolumn{1}{c}{14} & \multicolumn{1}{c}{28} & 
                \multicolumn{1}{c}{14} & \multicolumn{1}{c}{28} \\

            \hline

            Batch 1 & 5.62 & 7.49  & 0.71 & 0.85 \\
            Batch 2 & 8.02 & 10.08 & 1.15 & 1.33 \\

            \hline
        \end{tabular} 
        \begin{tablenotes}
            \item Results of the compressive and tensile tests at the end of 
            14 and 28 days of curing.
        \end{tablenotes}
    \end{threeparttable}
\end{table}

\begin{table}[!htb]
    \begin{threeparttable}
        \caption{\textbf{Permeability}}
        \label{table:perm-test}
        \setlength{\tabcolsep}{16.5pt}
        \def\arraystretch{1.5}%
        \begin{tabular}{ l r r }
            \hline
            & \multicolumn{1}{c}{14} & \multicolumn{1}{c}{28} \\
            \hline

            Batch 1 & 2.90 & 1.73 \\ 
            Batch 2 & 1.93 & 1.15 \\

            \hline
        \end{tabular} 
        \begin{tablenotes}
            \item Results of the permeability ($mm/s$) tests at the end of 
            14 and 28 days of curing.
        \end{tablenotes}
    \end{threeparttable}
\end{table}

From the results given by table \ref{table:str-test}, it is apparent that the 
amount of water used in the mix greatly affects the final strength of the 
concrete, with the ones having a lower water-cement ratio outperforming the ones 
with a higher water-cement ratio.
The results of the permeability tests shown by table \ref{table:perm-test} imply
that the permeability decreases with the increase in the water content of the 
mix. 


\section{Conclusion}
Pervious concrete is shaping up to be a promising solution to the serious
environmental caused by the lack of water reclamation by the soil due to the
impervious nature of regular concrete, something that paves a large portion of
our land. However, the use of pervious concrete is, as of now, limited to
pavements and parking lots, due to its relatively low strength. 
From the tests of the two mixes, it can be seen that strength and permeability
are inversely proportional to each other, and that the optimal mix should find
a balance between the two. The mix proposed in this paper, although not by any
means optimal, comes somewhat close to achieving this, performing reasonably
well in permeability, as well as strength tests.

\begin{thebibliography}{00}

\bibitem{b1} G. O. Young, ``Synthetic structure of industrial plastics,'' in \emph{Plastics,} 2\textsuperscript{nd} ed., vol. 3, J. Peters, Ed. New York, NY, USA: McGraw-Hill, 1964, pp. 15--64.

\bibitem{b2} W.-K. Chen, \emph{Linear Networks and Systems.} Belmont, CA, USA: Wadsworth, 1993, pp. 123--135.

\bibitem{b3} J. U. Duncombe, ``Infrared navigation---Part I: An assessment of feasibility,'' \emph{IEEE Trans. Electron Devices}, vol. ED-11, no. 1, pp. 34--39, Jan. 1959, 10.1109/TED.2016.2628402.

\bibitem{b4} E. P. Wigner, ``Theory of traveling-wave optical laser,'' \emph{Phys. Rev}., vol. 134, pp. A635--A646, Dec. 1965.

\bibitem{b5} E. H. Miller, ``A note on reflector arrays,'' \emph{IEEE Trans. Antennas Propagat}., to be published.

\bibitem{b6} E. E. Reber, R. L. Michell, and C. J. Carter, ``Oxygen absorption in the earth's atmosphere,'' Aerospace Corp., Los Angeles, CA, USA, Tech. Rep. TR-0200 (4230-46)-3, Nov. 1988.

\bibitem{b7} J. H. Davis and J. R. Cogdell, ``Calibration program for the 16-foot antenna,'' Elect. Eng. Res. Lab., Univ. Texas, Austin, TX, USA, Tech. Memo. NGL-006-69-3, Nov. 15, 1987.

\bibitem{b8} \emph{Transmission Systems for Communications}, 3\textsuperscript{rd} ed., Western Electric Co., Winston-Salem, NC, USA, 1985, pp. 44--60.

\bibitem{b9} \emph{Motorola Semiconductor Data Manual}, Motorola Semiconductor Products Inc., Phoenix, AZ, USA, 1989.

\bibitem{b10} G. O. Young, ``Synthetic structure of industrial
plastics,'' in Plastics, vol. 3, Polymers of Hexadromicon, J. Peters,
Ed., 2\textsuperscript{nd} ed. New York, NY, USA: McGraw-Hill, 1964, pp. 15-64.
[Online]. Available:
\underline{http://www.bookref.com}.

\bibitem{b11} \emph{The Founders' Constitution}, Philip B. Kurland
and Ralph Lerner, eds., Chicago, IL, USA: Univ. Chicago Press, 1987.
[Online]. Available: \underline{http://press-pubs.uchicago.edu/founders/}

\bibitem{b12} The Terahertz Wave eBook. ZOmega Terahertz Corp., 2014.
[Online]. Available:
\underline{http://dl.z-thz.com/eBook/zomegaebookpdf\_1206\_sr.pdf}. Accessed on: May 19, 2014.

\bibitem{b13} Philip B. Kurland and Ralph Lerner, eds., \emph{The
Founders' Constitution.} Chicago, IL, USA: Univ. of Chicago Press,
1987, Accessed on: Feb. 28, 2010, [Online] Available:
\underline{http://press-pubs.uchicago.edu/founders/}

\bibitem{b14} J. S. Turner, ``New directions in communications,'' \emph{IEEE J. Sel. Areas Commun}., vol. 13, no. 1, pp. 11-23, Jan. 1995.

\bibitem{b15} W. P. Risk, G. S. Kino, and H. J. Shaw, ``Fiber-optic frequency shifter using a surface acoustic wave incident at an oblique angle,'' \emph{Opt. Lett.}, vol. 11, no. 2, pp. 115--117, Feb. 1986.

\bibitem{b16} P. Kopyt \emph{et al., ``}Electric properties of graphene-based conductive layers from DC up to terahertz range,'' \emph{IEEE THz Sci. Technol.,} to be published. DOI: 10.1109/TTHZ.2016.2544142.

\bibitem{b17} PROCESS Corporation, Boston, MA, USA. Intranets:
Internet technologies deployed behind the firewall for corporate
productivity. Presented at INET96 Annual Meeting. [Online].
Available: \underline{http://home.process.com/Intranets/wp2.htp}

\bibitem{b18} R. J. Hijmans and J. van Etten, ``Raster: Geographic analysis and modeling with raster data,'' R Package Version 2.0-12, Jan. 12, 2012. [Online]. Available: \underline {http://CRAN.R-project.org/package=raster}

\bibitem{b19} Teralyzer. Lytera UG, Kirchhain, Germany [Online].
Available:
\underline{http://www.lytera.de/Terahertz\_THz\_Spectroscopy.php?id=home}, Accessed on: Jun. 5, 2014.

\bibitem{b20} U.S. House. 102\textsuperscript{nd} Congress, 1\textsuperscript{st} Session. (1991, Jan. 11). \emph{H. Con. Res. 1, Sense of the Congress on Approval of}  \emph{Military Action}. [Online]. Available: LEXIS Library: GENFED File: BILLS

\bibitem{b21} Musical toothbrush with mirror, by L.M.R. Brooks. (1992, May 19). Patent D 326 189 [Online]. Available: NEXIS Library: LEXPAT File: DES

\bibitem{b22} D. B. Payne and J. R. Stern, ``Wavelength-switched pas- sively coupled single-mode optical network,'' in \emph{Proc. IOOC-ECOC,} Boston, MA, USA, 1985, pp. 585--590.

\bibitem{b23} D. Ebehard and E. Voges, ``Digital single sideband detection for interferometric sensors,'' presented at the \emph{2\textsuperscript{nd} Int. Conf. Optical Fiber Sensors,} Stuttgart, Germany, Jan. 2-5, 1984.

\bibitem{b24} G. Brandli and M. Dick, ``Alternating current fed power supply,'' U.S. Patent 4 084 217, Nov. 4, 1978.

\bibitem{b25} J. O. Williams, ``Narrow-band analyzer,'' Ph.D. dissertation, Dept. Elect. Eng., Harvard Univ., Cambridge, MA, USA, 1993.

\bibitem{b26} N. Kawasaki, ``Parametric study of thermal and chemical nonequilibrium nozzle flow,'' M.S. thesis, Dept. Electron. Eng., Osaka Univ., Osaka, Japan, 1993.

\bibitem{b27} A. Harrison, private communication, May 1995.

\bibitem{b28} B. Smith, ``An approach to graphs of linear forms,'' unpublished.

\bibitem{b29} A. Brahms, ``Representation error for real numbers in binary computer arithmetic,'' IEEE Computer Group Repository, Paper R-67-85.

\bibitem{b30} IEEE Criteria for Class IE Electric Systems, IEEE Standard 308, 1969.

\bibitem{b31} Letter Symbols for Quantities, ANSI Standard Y10.5-1968.

\bibitem{b32} R. Fardel, M. Nagel, F. Nuesch, T. Lippert, and A. Wokaun, ``Fabrication of organic light emitting diode pixels by laser-assisted forward transfer,'' \emph{Appl. Phys. Lett.}, vol. 91, no. 6, Aug. 2007, Art. no. 061103.~

\bibitem{b33} J. Zhang and N. Tansu, ``Optical gain and laser characteristics of InGaN quantum wells on ternary InGaN substrates,'' \emph{IEEE Photon. J.}, vol. 5, no. 2, Apr. 2013, Art. no. 2600111

\bibitem{b34} S. Azodolmolky~\emph{et al.}, Experimental demonstration of an impairment aware network planning and operation tool for transparent/translucent optical networks,''~\emph{J. Lightw. Technol.}, vol. 29, no. 4, pp. 439--448, Sep. 2011.

\end{thebibliography}


\EOD

\end{document}
